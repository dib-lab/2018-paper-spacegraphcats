\subsection*{Efficient graph algorithms provide novel tools for investigating graph neighborhoods}

%This probably belongs in the introduction.
Recent work has shown that incorporating the structure of the assembly graph into the analysis of metagenome
data can provide a more complete picture of gene content \cite{perchlorate,metacherchant}. While this has provided evidence that it is
useful to analyze sequence that has small graph distance from a query (is in a ``neighborhood''), this approach
has not been widely adopted.
Na\"ively, local expansion around many queries in the assembly graph does not scale to these types of analyses
due to the overhead associated with searching
in a massive graph. The neighborhood index structure described in this work overcomes this computational
obstacle and enables rapid exploration of sequence data that is local to a query.

%In the past, biologists manually crafted genomic neighborhoods.
%SGC simplifies this dramatically by giving them a tool to query for neighborhoods of genomic sequences.
Because a partition into \pieces provides an implicit data reduction (the cDBG edge relationships are subsumed by \piece membership), the query-independent nature of the index allows many queries to be processed quickly without loading the entire graph into memory.
Our approach consequently provides a data exploration framework not otherwise available.

Exploiting the structural sparsity of cDBGs is a crucial component of our
algorithms. First, it is necessary to use graph structure to obtain a guarantee
that \algoref{alg:index_pieces} finds a small number of \pieces since the size
of a minimum $r$-dominating set cannot be approximated better than a factor of
$\log n$ in general graphs\footnote{That is, graphs about which we make no
structural assumptions.} unless $\text{NP}\subseteq \text{DTIME}(n^{O(\log \log
n)})$~\cite{chlebik2008approximation}. Without such a guarantee, we cannot be
sure that we are achieving significant data reduction by grouping cDBG vertices
into \pieces. Being able to do this in linear time also ensures that
indexing and querying can scale to very large data sets. Furthermore, because we
utilize a broad structural characterization (bounded expansion) of cDBGs rather than a
highly specialized aspect, our methods enable neighborhood-based
information retrieval in any domain whose graphs exhibit bounded expansion
structure; examples include some infrastructure,
social, and communication networks~\cite{demaine2014structural,felixThesis,wcol2018}.


% Novelty:
% * We create an index structure. k-mer -> piece -> bag of k-mers which is query-independent.
% * Inherent data reduction (neighborhoods are much smaller than the entire graph); we don't
% require you to load the entire graph in memory to do a single BFS. We only require you to load the
% neighborhoods.
%
% Theoretical Underpinning:
% * structural sparsity/efficient on sparse graphs
% * We run all experiments here with r = 1 (the simplest case; this already enables enough data reduction).
%
% Implementation:
% * There were several challenges in creating scalable effective algorithms and software.
% * summarize runtimes reported in results/supplement.
%
% Usability:
% * runtimes are reasonable relative to other analyses; scalability and memory usage.
% * Easy for biologists to use; integration with other pipelines? Designed as both Python API + standalone command-line tool.
% * You can explore your data by running lots of queries quickly. This isn't plausible in other paradigms.

\subsection*{Neighborhood queries extend genome bins}

In both the \podarv and \hu metagenomes, neighborhood queries were
able to identify additional content likely belonging to query genomes (@supportme).
In the \podarv mock metagenome, we retrieved a potentially complete
genome for an unknown strain based on partial matches to known
genomes. In the \hu metagenome, we increased the estimated
completeness of all but one genome bin -- in some cases substantially,
e.g. in the case of {\em Microgenomates bacterium 39\textunderscore 7}
we added an estimated 16.5\% to the genome bin.  In both cases we rely
solely on the structure of the assembly graph to expand the genome
bins. We do not make use of sequence composition, contig abundance, or
phylogenetic marker genes in our search. Thus graph proximity provides
an orthogonal set of information for genome-resolved metagenomics that
could be used to improve current binning techniques.

\subsection*{Query neighborhoods from real metagenomes contain substantial strain variation that may block assembly}

Previous work suggests that metagenome assembly and binning
approaches are fragile to strain variation \cite{CAMI,Awad155358}. This may
prevent the characterization of some genomes from metagenomes.  The extent of this problem is unknown, although
the majority of approaches to genome-resolved metagenomics rely on
assembly and thus could be affected.

In this work, we find that some of the sequence missing from genome
bins can be retrieved using neighborhood queries.  For \hu, some
genome bins are missing as many as 66.6\% of marker genes from the
original bins, with more than half of the 22 bins missing 20\% or
more; this accords well with evidence from a recent comparison of
single-cell genomes and metagenome-assembled genomes \cite{baltic}, in
which it was found that metagenome-assembled genomes were often
missing 20\% to 40\% of single-cell genomic sequence.  Neighborhood
query followed by amino acid assembly recovers additional content for
all but two of the genome bins; this is likely an underestimate, since
\plass may also be failing to assemble some content.

When we bioinformatically analyze the
function of the expanded genome content from neighborhood queries, our
results are consistent with the previous metabolic analyses by \cite{Hu2016}, and extend the
set of available genes by 13\%.
This suggests that current approaches to genome binning are specific,
and that the main question is sensitivity, which agrees with a more direct
measurement of lost content \cite{baltic}.

\subsection*{Neighborhood queries enable a genome-targeted workflow to recover strain variation}

The \sgc analysis workflow described above starts with genome bins. The
genome bins are used as a query into the metagenome assembly graph,
following which we extract reads from the query neighborhood.  We
assemble these reads with the \plass amino acid assembler, and then
analyze the assembly for gene content. We show that the \plass
assembly contains strain-level heterogeneity at the amino acid level,
and that this heterogeneity is supported by underlying nucleotide
diversity.  Even with stringent error trimming on the underlying
reads, we identify at least thirteen novel gyrA sequences in ten
genome neighborhoods.

Of note, this workflow explicitly associates the \plass assembled
proteins with specific genome bins, as opposed to a whole-metagenome
\plass assembly which recovers protein sequence from the entire
metagenome but does not link those proteins to specific genomes. The
binning-based workflow connects the increased sensitivity of \plass
assembly to the full suite of tools available for genome-resolved
metagenome analysis, including phylogenomic and metabolic
analysis. However, \sgc does not separate regions of the graph shared
in multiple query neighborhoods; existing
strain recovery approaches such as DESMAN or MSPminer could be used for this purpose
\cite{desman,mspminer}.

One future step could be to characterize unbinned genomic
content from metagenomes by looking at \plass-assembled marker genes
in the metagenome that do not belong to any bin's query neighborhood.
This would provide an estimate of the extent of
metagenome content remaining unbinned.
