\subsection*{Data}

We use two data sets: SRR606249 from
\texttt{podarV}~\cite{shakya2013comparative} and SRR1976948 (sample
SB1) from \texttt{hu}~\cite{hu}. Each data set was first preprocessed
to remove low-abundance k-mers as in \cite{Zhang2015}, using {\tt
  trim-low-abund.py} from khmer v2.1.2 \cite{Standage2017} with the
parameters {\tt -C 3 -Z 18 -M 20e9 -V -k 31}. We build compact De Bruijn
graphs using BCALM v2.2.0 \cite{bcalm}.  Stringent read trimming at
low-abundance k-mers was done with {\tt trim-low-abund.py} from khmer, with the
parameters {\tt -C 5 -M 20e9 -k 31}.

\subsection*{Software}

The source code for the index construction and search is available at
\url{https://github.com/spacegraphcats/spacegraphcats}~\cite{spacegraphcats}.
It is implemented in Python 3 under the 3-Clause BSD License.

Snakemake \cite{snakemake} workflows to reproduce all of the analysis
are available at
\url{https://github.com/dib-lab/2018-paper-spacegraphcats/}, and
Jupyter Notebooks to recreate Figures 1, 2, 3, and 4(b) are in that same
repository \cite{jupyter}. The notebooks rely on the numpy,
matplotlib, pandas, scipy, and Vega-Lite libraries
\cite{numpy,matplotlib,pandas,scipy,vegalite}.

\subsection*{Benchmarking}

We measured time and memory usage for Algorithms 1-3 by executing the
following targets in the \sgc \texttt{conf/Snakefile}:
\texttt{catlas.csv} for \texttt{rdomset}, \texttt{contigs.fa.gz.mphf}
for \texttt{indexPieces}, and \texttt{search} for \texttt{search}.  We
report wall time and maximum resident set size, running under Ubuntu
18.04 on an NSF Jetstream virtual machine with 10 cores and 30 GB of
RAM \cite{jetstream1,jetstream2}. To measure maximum resident set
size, we used the memusg script (Jaeho Shin,
https://gist.github.com/netj/526585).

\subsection*{Graph denoising}

For each data set, we built a compact De Bruijn graph (cDBG) for
k=31 by computing the set of unitigs with
BCALM~\cite{chikhi2016compacting}, removing all vertices of degree one
with a mean $k$-mer abundance of 1.1 or less, and then contracting
long degree-two paths when possible.

\subsection*{Neighborhood indexing and search}

We used \sgc to build an $r$-dominating set for each denoised cDBG and
index it. We then performed neighborhood queries with \sgc, which
produces a set of cDBG nodes and reads that contributed to them.  The
full list of query genomes for the {\em Proteiniclasticum} query is
  available in Supp Material \ref{subsec:query_accessions}, and
  the NCBI accessions for the {\em P. gingivalis}, {\em T. denticola}, 
and {\em B. thetaiotamicron} queries are in the directory {\tt pipeline-base} 
of the paper repository, files {\tt strain-gingivalis.txt}, 
{\tt strain-denticola.txt}, and {\tt strain-bacteroides.txt}, respectively.

\subsection*{Search results analysis}

Query neighborhood size, Jaccard containment, and Jaccard similarity
were estimated using modulo hash signatures with a k-mer size of 31 and
a scaled factor of 1000, as implemented in sourmash v2.0a9
\cite{sourmash}.

\subsection*{Assembly and genome bin analysis}

We assembled reads using MEGAHIT v1.1.3 \cite{megahit} and \plass
v2-c7e35 \cite{plass}, treating the reads as single-ended. Bin
completeness was estimated with CheckM 1.0.11, with the {\tt
  --reduced\_tree} argument \cite{CheckM}.  Amino acid identity
between bins and genomes was calculated using CompareM commit
7cd51276 (https://github.com/dparks1134/CompareM).

\subsection*{Gene targeted analysis}

Analysis of specific genes was done with HMMER v3.2.1
\cite{hmmer}. \plass amino acid assemblies were queried with HMMER {\tt hmmscan}
 using the PFAM domains listed in Supp Material \ref{tab:pfam_accessions}, 
using a threshold score of 100 \cite{pfam}. Matching sequences were then
extracted from the assemblies for further analysis. To overcome
problems associated with comparing non-overlapping sequence fragments,
only sequences that overlapped 125 of the most-overlapped 200 residues
of the PFAM domain were retained (all sequences shared a minimum
overlap of 50 residues with all other sequences). These sequences were
aligned with MAFFT v7.407 with the {\tt --auto} argument \cite{mafft}. Pairwise similarities were
calculated using HMMER where the final value represented the number of
identical amino acids in the alignment divided by the number of
overlapping residues between the sequences. Pairwise distances were
visualized using a multidimensional scaling calculated in R using the
{\tt cmdscale} function. To visualize the assembly graph structure
underlying these amino acid assemblies, we used paladin v1.3.1 
to map abundance-trimmed reads back to the \plass amino acid assembly,
with {\tt -f 125} to set the minimum ORF length accepted \cite{paladin}. 
We extracted the reads that mapped to the gene of interest, created an
assembly graph using BCALM v2.2.0 \cite{chikhi2016compacting}, and
visualized the graph using Bandage v0.8.1 \cite{bandage}. We colored
nucleotide sequences originating from the bins using the BLAST feature
in Bandage.

\subsection*{KEGG Analysis}

We annotated the \plass assemblies using Kyoto Encyclopedia of Genes
GhostKOALA v2.0 \cite{kegg}. To assign KEGG ortholog function, we used
methods implemented at https://github.com/edgraham/GhostKoalaParser
release 1.1.
