The neighborhood query approach described in this work provides an
alternative window into metagenome content associated with binned genomes. We extend previous work
showing that assembly-based methods are fragile to strain
variation, and provide an alternative workflow that substantially
broadens our ability to characterize metagenome content.  This first
investigation focuses on only two data sets, one mock and one real, but
the neighborhood indexing approach is broadly applicable to
all shotgun metagenomes.

In this initial investigation of neighborhood indexing, we have
focused on using neighborhood queries with a genome bin.  We recognize
that this approach is of limited use in regions where no genome bin is
available; \sgc is flexible and performant enough to support
alternative approaches such as querying with k-mers belonging to genes
of interest.

Potential applications of \sgc in metagenomics include developing
metrics for genome binning quality, analyzing pangenome neighborhood
structure, exploring $r$-dominating sets for $r > 1$, extending analyses
to colored De Bruijn graphs, and investigating {\em de novo}
extraction of genomes based on neighborhood content.  We could also
apply \sgc to analyze the neighborhood structure of assembly graphs
overlayed with physical contact information (from \eg HiC), which could
yield new applications in both metagenomics and genomics
\cite{Marbouty2014,Beitel2014}.

More generally, the graph indexing approach developed here may be
applicable well beyond metagenomes and sequence analysis.  The
exploitation of bounded expansion to efficiently compute $r$-dominating
sets on large graphs makes this technique applicable to a broad array
of problems.
