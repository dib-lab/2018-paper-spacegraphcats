\dropcap{M}etagenomics is the analysis of microbial communities through shotgun
DNA sequencing, which randomly samples many subsequences (reads)
from the genomic DNA of each microbe present in the community \cite{Quince2017}.

A common problem in metagenomics is the reconstruction of
individual microbial genomes from the mixture.
Typically this is
done by first running an assembly algorithm that reconstructs
longer linear regions based on a graph of the sampled
subsequences~\cite{pell2012scaling}, and then binning assembled
contigs together using compositional features and gene content~\cite{laczny2017busybee,lin2016accurate}.  These
``metagenome-assembled genomes'' are then
analyzed for phylogenetic markers and metabolic function. In recent years,
nearly 200,000 metagenome-assembled genomes have been published,
dramatically expanding our view of microbial life
\cite{Parks2017,Tully2018,Stewart2018,Delmont2018,Hug2016,Pasolli2019}.

Information present in shotgun metagenomes is often omitted from the
binned genomes due to either a failure to
assemble \cite{CAMI,Awad155358} or a failure to bin.  The underlying
technical reasons for these failures include low coverage, high
sequencing error, high strain variation, and/or sequences with
insufficient compositional or genic signal.  Recent work has
particularly focused on the problem of strain confusion, in which high
strain variation results in considerable fragmentation of assembled
content in mock or synthetic communities \cite{CAMI,Awad155358}; the
extent and impact of strain confusion in real metagenomes is still
unknown but potentially significant - metagenome-assembled genomes may be missing 20-40\% of true content \cite{brownstrain,Brito2016,baltic}.\looseness-1

%TER "fragmentation of assembled content" -- strain variation breaks the assembly, which results in smaller contigs that cannot be binned.
Associating unbinned metagenomic sequence to inferred bins or known
genomes is technically challenging.  Some approaches use mapping or
k-mer baiting, in which assembled sequences are used to extract reads
or contigs from a metagenome or
graph \cite{desman,Nayfach2016,ekg,mspminer,Petersen2016}.
These
methods fail to recover genomic content that does not directly overlap
with the query, such as large indels or novel genomic
islands. Moreover, most assembly graphs undergo substantial heuristic
error pruning and may not contain relevant content
\cite{CAMI,Awad155358}.
Graph queries have shown promise for recovering sequence from regions that do
not assemble well but are graph-proximal to the query \cite{metacherchant,perchlorate}. However, many graph query
algorithms are NP-hard and hence computationally intractable in the
general case; compounding the computational challenge, metagenome assembly
graphs are frequently large, with millions of nodes, and require 10s
to 100s of gigabytes of RAM for storage.\looseness-1

In this paper, we develop and implement a scalable graph query
framework for extracting unbinned sequence from metagenome assembly
graphs with millions of nodes. Crucially, we exploit the structural
sparsity of compact De Bruijn assembly graphs in order to compute a
succinct index data structure in linear time. Our initial
investigations presented here focus on using this index to perform
neighborhood queries in large assembly graphs to investigate genome
binning and content recovery.  This enables us to extract the genome
of a novel bacterial species, recover missing sequence variation in
amino acid sequences for genome bins, and identify missing content for
metagenome-assembled genomes.  Our query methods are assembly-free and
avoid techniques that may discard strain information such as error
correction.  These algorithms are available in an open-source Python
software package,
\textsf{spacegraphcats}~\cite{spacegraphcats}.

% Add read graph mapping.
